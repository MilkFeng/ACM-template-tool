%==============================常用宏包、环境==============================%
\documentclass[twocolumn,4paper]{article}
\usepackage{xeCJK} % For Chinese characters
\usepackage[hidelinks]{hyperref} % 目录超链接
\usepackage{amsmath, amsthm}
\usepackage{listings,xcolor}
\usepackage{geometry} % 设置页边距
\usepackage{fontspec}
\usepackage{graphicx}
\usepackage{fancyhdr} % 自定义页眉页脚
\setsansfont{Consolas} % 设置英文字体
\setmonofont[Mapping={}]{Consolas} % 英文引号之类的正常显示,相当于设置英文字体
\geometry{left=1.5cm,right=1.5cm,top=2cm,bottom=1cm} % 页边距
\setlength{\columnsep}{30pt}
% \setlength\columnseprule{0.4pt} % 分割线
%==============================常用宏包、环境==============================%

%==============================页眉、页脚、代码格式设置==============================%
% 页眉、页脚设置
\pagestyle{fancy}
\lhead{ACM/ICPC模板}
\rhead{第\thepage 页}
\lfoot{}
\cfoot{}
\rfoot{}
\renewcommand{\headrulewidth}{0.4pt}

% 代码格式设置
\lstset{
    language    = c++,
    numbers     = left,
    numberstyle = \small,
    breaklines  = true,
    captionpos  = b,
    tabsize     = 4,
    frame       = single,
    commentstyle = \color[RGB]{0,128,0},
    keywordstyle = \color[RGB]{0,0,255},
    basicstyle   = \small\ttfamily,
    stringstyle  = \color[RGB]{148,0,209}\ttfamily,
    rulesepcolor = \color{red!20!green!20!blue!20},
    showstringspaces = false,
}
%==============================页眉、页脚、代码格式设置==============================%

%==============================标题和目录==============================%
\title{\CJKfamily{hei} \bfseries ACM/ICPC模板}
\author{XXXXXX}
\renewcommand{\today}{\number\year 年 \number\month 月 \number\day 日}

\begin{document}
	\begin{titlepage}
		\maketitle
		\thispagestyle{empty}
	\end{titlepage}
    \renewcommand{\contentsname}{目录}
    \tableofcontents
    \thispagestyle{empty}
    \newpage
    %==============================标题和目录==============================%
    
    %==============================正文部分==============================%
    \onecolumn
    \setcounter{page}{1}   %new page
    \newpage
    \section{图论}
        \subsection{Dijkstra}
            \subsubsection{堆优化的Dijkstra}
                \lstinputlisting[language=C++]{./code/图论/Dijkstra/堆优化的Dijkstra.cpp}
            \subsubsection{朴素的Dijkstra}
                \lstinputlisting[language=C++]{./code/图论/Dijkstra/朴素的Dijkstra.cpp}
    \newpage
    \section{字符串}
        \subsection{KMP}
            \lstinputlisting[language=C++]{./code/字符串/KMP.cpp}
        \subsection{后缀数组}
            \lstinputlisting[language=C++]{./code/字符串/后缀数组.cpp}

%==============================正文部分==============================%
\end{document}